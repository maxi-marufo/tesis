\chapter{Calibración}\label{appendix:calibracion} 

En problemas de clasificación, el subproblema de la predicción de estimaciones
de probabilidad representativas de las probabilidades verdaderas es conocido
como calibración. En los sistemas del mundo real, los clasificadores no sólo
deben ser precisos, sino que también deben indicar cuando es probable que sean
incorrectos. Es decir, deben estimar su nivel de incertidumbre o confiabilidad.
En otras palabras, las probabilidades asociadas con la etiquetas de clase
predichas deben reflejar su verosimilitud real.

Uno de los casos en donde se debe contemplar este problema es en la toma de
decisiones (es decir, en casi toda aplicación real). Por ejemplo, en sistemas
utilizados para la salud, un diagóstico con un bajo nivel de confianza puede
significar realizar otro tipo de chequeo. A su vez, las estimaciones de
probabilidad, pueden ser utilizadas para ser incorporadas en otro modelo
probabilístico. Por ejemplo, se podrían combinar distintas salidas de distintos
modelos de forma ponderada para obtener una predicción más robusta frente los
casos en donde cada modelo individual falla.

La mayoría de los métodos de aprendizaje supervisado producen clasificadores que
generan puntuaciones \(s(x)\) que se pueden utilizar para ranquear los ejemplos
en el conjunto de pruebas de la etiqueta más probable a la menos probable de una
clase \(c\). Es decir, para dos ejemplos \( \mathbf{x}_1\) y \( \mathbf{x}_2\),
si \(s( \mathbf{x}_1) < s( \mathbf{x}_2)\) entonces \(\mathbb{P}(c|x) <
\mathbb{P}(c|y)\). Sin embargo, la clasificación según el rango de la
probabilidad de pertenencia a una clase no es suficiente. Lo que se necesita es
una estimación precisa de la probabilidad de que cada ejemplo de prueba sea un
miembro de la clase de interés.

El modelo de regresión logística es un caso especial ya que está bien calibrado
por diseño dado que su función objetivo minimiza la función de pérdida
logarítmica o {\it log-loss\/}~\cite{morrison2013tutorial}. Sin  embargo, si no
se cuenta con un conjunto de entrenamiento lo suficientemente grande, es posible
que el modelo no tenga suficiente información para calibrar las probabilidades.

Otros modelos, en cambio, no presentan esta propiedad (por ejemplo, los
clasificadores de Naive Bayes, Random Forest o redes
neuronales~\cite{zadrozny2002transforming, niculescu2005predicting,
guo2017calibration}). Incluso, hay modelos que no devuelven probabilidades {\it
a posteriori}, sino genéricas puntuaciones de confianza, como es el caso de
SVM~\cite{platt1999probabilistic}. En estos dos últimos casos es posible mapear
las salidas de los clasificadores a probabilidades {\it a posteriori\/}
calibradas a través de algunos método de
calibración~\cite{platt1999probabilistic, zadrozny2002transforming,
niculescu2005predicting, guo2017calibration}.

\section{Definición}

\section{Diagramas de confiabilidad}

\section{Métodos de Evaluación}

\section{Métodos de Calibración}

\subsection{Modelos Binarios}

\subsection{Modelos Multiclase}