\chapter{Experimentos}\label{experimentos}

En esta sección se evaluarán todos los métodos mencionados en el
Capítulo~\ref{estimacion} mediante nuevos casos de ejemplo simulados. Dicha
evaluación se basará en las métricas elegidas en~\ref{evaluacion:eleccion} (RAE,
AE y SE) además del sesgo. Además, la simulación se basará en el protocolo {\it
APP\/} definido en~\ref{evaluacion:protocolos}. No se realizará el proceso de
selección de modelos mencionado en~\ref{evaluacion:seleccion} ya que el objetivo
de estos experimentos no es el de buscar los mejores cuantificadores para los
casos de ejemplo, sino hacer una comparación de los mismos frente a iguales
condiciones (hiperparámetros por defecto, mismo clasificador, etc.). Finalmente,
se elaboran conclusiones en base a los resultados y se comparan con los
resultados de otros trabajos~\cite{moreo2021re, moreo2022tweet, moreo2021quapy,
tasche2016does, schumacher2021comparative}.

\section{Simulación}\label{experimentos:simulacion}

\subsection{Poblaciones}\label{experimentos:poblaciones}

La simulación consiste, en primer lugar, en generar dos poblaciones sintéticas
de datos. Al igual que en~\ref{estimacion:ejemplo}, para sintetizar los datos se
utilizó el algoritmo propuesto por~\citet{Guyon2003DesignOE} mediante la
función~\href{https://scikit-learn.org/stable/modules/generated/sklearn.datasets.make_classification.html}{{\it
make\_classification}} de {\it scikit-learn}. En este caso, se crearon dos
poblaciones, {\it F\/} (fácil) y {\it D\/} (difícil), cuyos argumentos de
creación son (los parámetros no mencionados usan sus valores por defecto):
\begin{center}
    \begin{tabular}{l|ll|ll}
        \cline{2-3}
        & \multicolumn{2}{l|}{Población} & & \\ \cline{2-3} &
        \multicolumn{1}{l|}{F}     & D     & &  \\ \cline{1-3}
        \multicolumn{1}{|l|}{n\_samples}
        & \multicolumn{1}{l|}{51000} & 51000 &  &  \\
        \multicolumn{1}{|l|}{n\_features}    & \multicolumn{1}{l|}{2}     & 100
        & &  \\
        \multicolumn{1}{|l|}{n\_informative} & \multicolumn{1}{l|}{2}     & 5 &
        &  \\
        \multicolumn{1}{|l|}{n\_redundant}   & \multicolumn{1}{l|}{0}     & 15 &
        &  \\
        \multicolumn{1}{|l|}{n\_repeated}    & \multicolumn{1}{l|}{0}     & 15 &
        &  \\
        \multicolumn{1}{|l|}{flip\_y}        & \multicolumn{1}{l|}{0}     & 0 &
        &  \\
        \multicolumn{1}{|l|}{class\_sep}     & \multicolumn{1}{l|}{0.6}   & 0.2
        &  &  \\ \cline{1-3}
    \end{tabular}
    \captionof{table}{Poblaciones}\label{experimentos:tabla_poblaciones}
\end{center}

La elección de estos valores tiene como objetivo generar una población fácil de
clasificar y con características de baja dimensionalidad, y otra más difícil y
con alta dimensionalidad, para probar bajo estas dos condiciones cada método.
Además, al usar los parámetros {\it n\_classes\/} y {\it weights\/} por defecto,
estaremos generando datasets binarios y perfectamente balanceados.

\subsection{Datasets}\label{experimentos:datasets}

Tenemos entonces dos poblaciones sintéticas, cada una de 51000 individuos.
Luego, procedemos a sub-dividir ambas poblaciones de forma estratificada (es
decir, manteniendo la proporción de clases balanceadas) para crear los datasets
de entrenamiento y de prueba en cada población. Dichos datasets son de 50000 y
1000 individuos respectivamente.

Siguiendo el protocolo {\it APP\/} ya mencionado, la simulación consistió en
muestrear de forma iterativa el dataset de entrenamiento de cada población con
distintos tamaños de muestra (500 y 5000), distintas prevalencias\footnote{Para
evitar repeticiones, cuando mencionemos prevalencia en estos casos se hará
referencia siempre a la clase positiva.} (0.01, 0.25, 0.5, 0.75 y 0.99) y de
forma repetida (5 veces) -es decir, un total de 50 muestras distintas-. Luego,
por cada muestra de entrenamiento, a su vez, se muestreo de forma iterativa el
dataset de evaluación. En este caso también se tomaron distintos tamaños de
muestra (10 y 100), distintas prevalencias (0, 0.2, 0.5, 0.8 y 1.0 para las
muestras de tamaño n=10, y 0.01, 0.25, 0.5, 0.75 y 0.99 para las muestras de
tamaño n=100) y de forma repetida (5 veces) -50 muestras de prueba por cada
muestra de entrenamiento-.
\begin{center}
    \begin{tabular}{l|cccc|}
        \cline{2-5}
        & \multicolumn{4}{c|}{Dataset} \\
        \cline{2-5} &
        \multicolumn{2}{c|}{Train}
        & \multicolumn{2}{c|}{Test} \\
        \hline
        \multicolumn{1}{|l|}{n\_samples}  & \multicolumn{1}{c|}{500} &
        \multicolumn{1}{c|}{5000}                    &
        \multicolumn{1}{c|}{10}
        & 100 \\ \hline
        \multicolumn{1}{|l|}{prev}        &
        \multicolumn{2}{c|}{\begin{tabular}[c]{@{}c@{}}0.01\\ 0.25\\ 0.5\\
        0.75\\ 0.99\end{tabular}} &
        \multicolumn{1}{c|}{\begin{tabular}[c]{@{}c@{}}0\\ 0.2\\ 0.5\\ 0.8\\
        1.0\end{tabular}} & \begin{tabular}[c]{@{}c@{}}0.01\\ 0.25\\ 0.5\\
        0.75\\ 0.99\end{tabular} \\ \hline
        \multicolumn{1}{|l|}{n\_repetitions} & \multicolumn{2}{c|}{5} &
        \multicolumn{2}{c|}{5} \\ \hline
    \end{tabular}
    \captionof{table}{Datasets}\label{experimentos:tabla_datasets}
\end{center}

\subsection{Cuantificación}\label{experimentos:cuantificacion}

Por cada muestra de entrenamiento, se procede a ajustar cada uno de los métodos
a evaluar. Empezando por los métodos agregativos que utilizan clasificadores
generales (\ref{estimacion:generales}), tenemos los que requieren como entrada
las etiquetas de las clases predichas (es decir, que usan clasificadores duros).
En nuestro caso, estos métodos son el {\it CC\/} y el {\it ACC}. Es decir, que
primero debemos ajustar un modelo de clasificación. En esta simulación hemos
utilizado dos modelos distintos, un modelo de regresión logística y el
clasificador de XGBoost, con la idea de probar un modelo simple y uno complejo,
con el objetivo de determinar si la complejidad del modelo mejora o no la
cuantificación. En este caso no es necesario calibrarlos ya que usaremos
solamente las etiquetas predichas. Para sus hiperparámetros usaremos los valores
por defecto de las librerías {\it scikit-learn\/} y {\it xgboost\/}
respectivamente.
\begin{center}
    \begin{tabular}{|lc|}
        \hline
        \multicolumn{2}{|c|}{Clasificación}                     \\ \hline
        \multicolumn{1}{|c|}{Modelo}             & Complejidad  \\ \hline
        \multicolumn{1}{|l|}{LogisticRegression} & Baja         \\
        \multicolumn{1}{|l|}{XGBoost}            & Alta         \\ \hline
    \end{tabular}
\captionof{table}{Modelos de
clasificación}\label{experimentos:tabla_clasificacion}
\end{center}

En cambio, para los métodos agregativos que utilizan clasificadores generales
pero que requieren como entrada las probabilidades {\it a posteriori\/} de
pertenencia a cada clase (es decir, clasificadores blandos), debemos no solo
ajustar los modelos de clasificación, sino que también conviene calibrarlos (ya
que es un supuesto de estos métodos). En nuestro caso, estos métodos son el {\it
PCC}, {\it PACC}, {\it EMQ\/} y {\it HDy}. Con respecto a los algoritmos de
calibración, utilizaremos los cuatro métodos para modelos binarios propuestos
por~\citet{guo2017calibration} y mencionados en el
Apéndice~\ref{appendix:calibracion}, además de probar también los clasificadores
sin calibrar. Como los métodos de calibración requieren utilizar validación
cruzada para ajustar sus parámetros, utilizaremos el método de {\it held-out},
es decir que destinaremos una porción (20\%) del dataset de entrenamiento para
ello.
\begin{center}
    \begin{tabular}{|lc|}
        \hline
        \multicolumn{2}{|c|}{Calibración}                    \\ \hline
        \multicolumn{1}{|c|}{Método}              & Held-out \\ \hline
        \multicolumn{1}{|l|}{No Calibration}      & 0\%      \\
        \multicolumn{1}{|l|}{Histogram Binning}   & 20\%     \\
        \multicolumn{1}{|l|}{Isotonic Regression} & 20\%     \\
        \multicolumn{1}{|l|}{BBQ}                 & 20\%     \\
        \multicolumn{1}{|l|}{Platt Scaling}       & 20\%     \\ \hline
    \end{tabular}
    \captionof{table}{Modelos de
    calibración}\label{experimentos:tabla_calibracion}
\end{center}

Para los métodos agregativos que utilizan clasificadores específicos
(\ref{estimacion:especificos}) -en nuestro caso, únicamente el método {\it
ELM\/}-, debemos también ajustar un modelo de clasificación previo a la
cuantificación, pero en este caso no utilizaremos los nombrados
en~\ref{experimentos:tabla_clasificacion}, sino uno optimizado para ser
utilizado en cuantificación. En este caso usaremos, al igual que en el
Capítulo~\ref{evaluacion}, el clasificador \({\it SVM \/}_{perf}\), pero esta
vez optimizado no solo para la métrica KLD, sino también AE y RAE.

Para los modelos no agregativos (\ref{estimacion:no_agregativos}), no se usan
modelos de clasificación, y por lo tanto tampoco se calibran; simplemente se
aplica directamente el método de cuantificación. En nuestro caso, para este
grupo usaremos solo el método {\it HDx}, ya que no usaremos modelos de
clasificación generativos explícitos. Sin embargo, sí usaremos el método {\it
LR-Implicit}, el cual, como bien dijimos, se debe considerar en realidad un
método agregativo, y así lo haremos en las simulaciones.

Cabe mencionar que para los métodos de cuantificación que requieren de
validación cruzada para estimar sus parámetros internos también utilizaremos el
método de {\it held-out}, destinando otra porción (20\%) del dataset de
entrenamiento para ello.
\begin{center}
    \begin{tabular}{|lc|}
        \hline
        \multicolumn{2}{|c|}{Cuantificación}                              \\
        \hline
        \multicolumn{1}{|c|}{Método}      & \multicolumn{1}{c|}{Held-out} \\
        \hline
        \multicolumn{1}{|l|}{CC}          & \multicolumn{1}{c|}{0\%}      \\
        \multicolumn{1}{|l|}{ACC}         & \multicolumn{1}{c|}{20\%}     \\
        \multicolumn{1}{|l|}{PCC}         & \multicolumn{1}{c|}{0\%}      \\
        \multicolumn{1}{|l|}{PACC}        & \multicolumn{1}{c|}{20\%}     \\
        \multicolumn{1}{|l|}{TH\_MAX}     & \multicolumn{1}{c|}{20\%}     \\
        \multicolumn{1}{|l|}{TH\_X}       & \multicolumn{1}{c|}{20\%}     \\
        \multicolumn{1}{|l|}{TH\_T50}     & \multicolumn{1}{c|}{20\%}     \\
        \multicolumn{1}{|l|}{TH\_MS}      & \multicolumn{1}{c|}{20\%}     \\
        \multicolumn{1}{|l|}{EMQ}         & \multicolumn{1}{c|}{0\%}      \\
        \multicolumn{1}{|l|}{HDy}         & \multicolumn{1}{c|}{20\%}     \\
        \multicolumn{1}{|l|}{HDx}         & \multicolumn{1}{c|}{0\%}      \\
        \multicolumn{1}{|l|}{ELM-SVMperfKLD}  & \multicolumn{1}{c|}{0\%} \\
        \multicolumn{1}{|l|}{ELM-SVMperfAE}   & \multicolumn{1}{c|}{0\%} \\
        \multicolumn{1}{|l|}{ELM-SVMperfRAE}  & \multicolumn{1}{c|}{0\%} \\
        \multicolumn{1}{|l|}{LR-Implicit} & \multicolumn{1}{c|}{0\%}      \\
        \hline
    \end{tabular}
    \captionof{table}{Modelos de
    cuantificación}\label{experimentos:tabla_cuantificacion}
\end{center}

Finalmente, una vez ajustado el método de cuantificación para una muestra de
entrenamiento, se procede a tomar una muestra de prueba (recordando que
probaremos con distintos tamaños de muestra, prevalencia y repitiendo el
muestreo) y estimar su prevalencia.

Para la implementación de los métodos de cuantificación en la simulación se
utilizó la librería \href{https://github.com/HLT-ISTI/QuaPy}{{\it
Quapy}}~\cite{moreo2021quapy} (\url{https://github.com/HLT-ISTI/QuaPy}), excepto
para el método \href{https://github.com/slanglab/freq-e}{{\it
LR-Implicit}}~\cite{keith2018uncertainty}, para el cual se empleó el código
provisto por los autores (\url{https://github.com/slanglab/freq-e}). Para la
implementación de los modelos de calibración utilizamos la
librería~\href{https://github.com/EFS-OpenSource/calibration-framework}{{\it
net:cal}}~\cite{kuppers2020multivariate}
(\url{https://github.com/EFS-OpenSource/calibration-framework}).

\subsection{Evaluación}\label{experimentos:evaluación}

Como ya se mencionó, se realizó la evaluación en base a las métricas elegidas
en~\ref{evaluacion:eleccion}, es decir, se utilizaron RAE, AE y SE. La
evaluación se hizo según distintos criterios, agrupando los resultados en base a
estos criterios y calculando su intervalo de confianza del 95\%. Además, en los
casos que corresponda, usaremos también la métrica de clasificación F1 y la
métrica de calibración ECE para elaborar conclusiones.

\section{Conclusiones}\label{experimentos:conclusiones}

Empezamos por analizar el rendimiento de los cuantificadores en general. En la
tabla~\ref{experimentos:by_quantifier} se muestran los resultados de las
estimaciones de las métricas RAE, AE y SE para cada cuantificador usado en la
simulación. Los métodos basados en selección de umbrales ({\it TH\/}), {\it
PACC\/}, {\it LR-Implicit\/} y {\it EMQ\/} son los que en general mejor
desempeño tienen, mientras que los basados en minimización de pérdida explícita
({\it ELM\/}) son los peores. Estos resultados son congruentes con los
de~\citet{schumacher2021comparative} (en donde {\it TH\_MS\/} es el método con
mejor AE), los de~\citet{moreo2021re} (que afirma que {\it PACC\/} es el mejor
método dentro de las variantes de {\it CC\/}), los de~\citet{moreo2021quapy}
(donde {\it EMQ\/} resultó el método con mejor AE), los
de~\citet{moreo2022tweet} (donde también {\it EMQ\/} y {\it PACC\/} resultaron
los mejores métodos evaluados con AE) y los de~\citet{tasche2016does} (que
concluye que los métodos {\it CC\/} y {\it PCC\/} son limitados frente a {\it
ACC\/} y {\it PAC\/}).

Si queremos ver si los métodos sobre o sub estiman las prevalencias, debemos
entonces analizar el sesgo. Además, en este caso debemos mirar la distribución
del mismo, ya que su estimación puntual podría dar baja por más que los métodos
tengan error alto (ya que pueden cancelarse entre sí). Podemos observar en la
figura~\ref{fig:global_bias_by_quantifier} que {\it HDy\/} tiende a subestimar,
mientras que el resto de los métodos no presentan, en general, un sesgo marcado.
En esta figura también se verifica que {\it TH\_MS} tiene el mejor
comportamiento. Sin embargo, si analizamos el sesgo según la verdadera
prevalencia en la muestra de prueba
(figura~\ref{fig:diagonal_by_cuantificator}), todos los métodos si tienen un
sesgo en favor de la clase minoritaria.

\begin{figure}[H]
    \centering
    \centerline{\includegraphics[width=0.8\textwidth]{../experiments/plots/global_bias_by_quantifier.png}}
    \caption{Sesgo por método de
    cuantificación}\label{fig:global_bias_by_quantifier}
\end{figure}
\begin{figure}[H]
    \centering
    \centerline{\includegraphics[width=0.6\textwidth]{../experiments/plots/diagonal_by_cuantificator.png}}
    \caption{Prevalencia estimada por método de cuantificación según prevalencia
    de prueba}\label{fig:diagonal_by_cuantificator}
\end{figure}

Por otro lado, también verificamos algunas de los resultados que eran de
esperarse:
\begin{itemize}
    \item Para los métodos de cuantificación agregativos con clasificadores
    generales (\ref{estimacion:generales}), a mejor desempeño en la
    clasificación (métrica F1), mejor desempeño en la cuantificación
    (tabla~\ref{experimentos:by_classifier}).
    \item A menor cantidad de características y mayor separación entre clases en
    las poblaciones, mejor rendimiento de cuantificación
    (tabla~\ref{experimentos:by_population}).
    \item A mayor tamaño en muestra de entrenamiento, mejor rendimiento de
    cuantificación (tabla~\ref{experimentos:by_train_sample_size}) (coincide con
    los resultados de~\citet{schumacher2021comparative}).
\end{itemize}

Si analizamos según el tamaño de la muestra de evaluación
(tabla~\ref{experimentos:by_test_sample_size}), teniendo que para cada tamaño
utilizamos distintas posibles prevalencias como definimos en la
tabla~\ref{experimentos:tabla_datasets}, y recordando que los valores de las
métricas dependen de los mínimos valores posibles de las prevalencias de prueba
(ecuaciones~\ref{evaluacion:eq_zae} y~\ref{evaluacion:eq_zrae}), entendemos
entonces la diferencia de resultados en las métricas entre los distintos tamaños
de muestra.

Vemos también que cuanto más balanceada sea la muestra de entrenamiento, mejores
resultados de cuantificación se obtienen (tabla~\ref{experimentos:by_train_prev}
y figuras~\ref{fig:global_bias_by_train_prev}
y~\ref{fig:diagonal_by_train_prev}). Este es un dato bastante interesante, ya
que implicaría que podríamos aplicar en cuantificación las mismas técnicas de
balance de clases que se usan en problemas de clasificación para tratar de
conseguir mejores resultados que si usáramos datos desbalanceados. También vemos
como la prevalencia de la muestra de entrenamiento influye en el sesgo del
cuantificador, tendiendo en general a sobrestimar la clase mayoritaria. En la
figura~\ref{fig:diagonal_by_quantifier_by_train_prev} podemos ver estos
resultados desagregados por método de cuantificación.

\begin{figure}[H]
    \centering
    \centerline{\includegraphics[width=0.6\textwidth]{../experiments/plots/global_bias_by_train_prev.png}}
    \caption{Sesgo por prevalencia de
    entrenamiento}\label{fig:global_bias_by_train_prev}
\end{figure}
\begin{figure}[H]
    \centering
    \centerline{\includegraphics[width=0.5\textwidth]{../experiments/plots/diagonal_by_train_prev.png}}
    \caption{Prevalencia estimada por prevalencia de entrenamiento según
    prevalencia de prueba}\label{fig:diagonal_by_train_prev}
\end{figure}

\begin{figure}[H]
    \centering
    \begin{subfigure}[b]{.4\textwidth}
        \centering
        \includegraphics[width=\linewidth]{../experiments/plots/diagonal_by_cuantificator_0.01.png}\quad
    \end{subfigure}
    \begin{subfigure}[b]{.4\textwidth}
        \centering
        \includegraphics[width=\linewidth]{../experiments/plots/diagonal_by_cuantificator_0.25.png}\quad
    \end{subfigure}
    \medskip
    \begin{subfigure}[b]{.4\textwidth}
        \centering
        \includegraphics[width=\linewidth]{../experiments/plots/diagonal_by_cuantificator_0.50.png}
    \end{subfigure}
    \begin{subfigure}[b]{.4\textwidth}
        \centering
        \includegraphics[width=\linewidth]{../experiments/plots/diagonal_by_cuantificator_0.75.png}\quad
    \end{subfigure}
    \begin{subfigure}[b]{.5\textwidth}
        \centering
        \includegraphics[width=\linewidth]{../experiments/plots/diagonal_by_cuantificator_0.99.png}
    \end{subfigure}
    \medskip
    \hfill
    \caption{Prevalencia estimada por método de cuantificación según prevalencia
    de prueba, para distintas prevalencias de
    entrenamiento}\label{fig:diagonal_by_quantifier_by_train_prev}
\end{figure}

Con respecto al balance en la muestra de prueba
(tabla~\ref{experimentos:by_test_prev}), debemos volver a mencionar el
comentario en referencia a los distintos valores de prevalencia para los
distintos posibles tamaños de muestra. De todas formas, llama la atención que
las métricas son mejores cuando el desbalance de clases lleva a la clase
positiva a ser la minoritaria. Esto se debe a que los métodos de cuantificación
binaria basados en la estimación del {\it fpr\/} y {\it tpr\/} ({\it ACC}, {\it
PACC}, y {\it TH\/}) no son simétricos en cuanto a las clases positivas y
negativas, sino que son sensibles a si la clase mayoritaria es una u otra.

Si tenemos en cuenta la diferencia absoluta entre las prevalencias de la muestra
de entrenamiento y la de prueba (tabla~\ref{experimentos:by_abs_dataset_shift}),
vemos que también en general (y como quizás era esperable de forma intuitiva), a
menor diferencia, mejor rendimiento. Sin embargo, llama la atención cuan
dinámico es este comportamiento según el método de cuantificación
(tabla~\ref{experimentos:by_abs_dataset_shift_and_quantificator}). En general,
los métodos que mejor funcionan cuando el {\it dataset shift\/} es bajo son los
que peor funcionan cuando es alto. A pesar de que en~\citet{moreo2022tweet}
y~\citet{moreo2021quapy} el método {\it EMQ\/} es el de mejor rendimiento para
los casos de mayor {\it dataset shift}, en nuestras simulaciones fue {\it
TH\_T50\/} el método con mejores métricas ante estos casos.

\begin{figure}[H]
    \centering
    \centerline{\includegraphics[width=0.5\textwidth]{../experiments/plots/global_bias_by_calibration.png}}
    \caption{Sesgo por método de
    calibración}\label{fig:global_bias_by_calibration}
\end{figure}
\begin{figure}[H]
    \centering
    \centerline{\includegraphics[width=0.6\textwidth]{../experiments/plots/diagonal_by_calibration.png}}
    \caption{Prevalencia estimada por método de calibración según prevalencia en
    muestra de prueba}\label{fig:diagonal_by_calibration}
\end{figure}

Analizando ahora la influencia de la calibración en la cuantificación, la
calibración por regresión isotónica lleva una leve ventaja en el rendimiento de
la cuantificación en general (considerando solo métodos de cuantificación
agregativos con clasificadores generales y blandos)
(tabla~\ref{experimentos:by_calibration} y
figuras~\ref{fig:global_bias_by_calibration} y
~\ref{fig:diagonal_by_calibration}), aunque llamativamente no presenta ventajas
en cuánto a la métrica de clasificación (F1) ni de calibración (ECE).

Si tenemos en cuenta tanto el método de cuantificación como el de calibración
(en los casos que aplique)
(tabla~\ref{experimentos:by_quantifier_and_calibration} y
figura~\ref{fig:global_bias_by_quantifier_and_calibration}), se observa que {\it
LR-Implicit}, {\it EMQ} y {\it PACC} en los casos de calibración isotónica y de
no calibración se vuelven más competitivos frente a los métodos {\it TH}.

\begin{figure}[H]
    \centering
    \centerline{\includegraphics[width=\textwidth]{../experiments/plots/global_bias_by_quantifier_and_calibration.png}}
    \caption{Sesgo por método de calibración y
    cuantificación}\label{fig:global_bias_by_quantifier_and_calibration}
\end{figure}

Sin embargo, también notamos que esta mejora en la cuantificación mediante la
calibración surge cuanto más desbalanceados estén los datos de entrenamiento, no
habiendo mejoras significativas cuando están balanceados
(tabla~\ref{experimentos:by_train_prev_and_calibration}).

