\chapter{Estimación Puntual}

Durante los últimos años, se han propuesto varios métodos de cuantificación
desde diferentes perspectivas y con diferentes objetivos. En términos generales,
se pueden distinguir dos grandes clases de métodos en la literatura. La primera
clase es la de métodos agregativos, es decir, métodos que requieren la
clasificación de todos los individuos como un paso intermedio. Dentro de los
métodos agregativos, se pueden identificar dos subclases. La primera subclase
incluye métodos basados en clasificadores de propósito general; en estos métodos
la clasificación de los elementos individuales realizados como un paso
intermedio puede lograrse mediante cualquier clasificador. La segunda subclase
se compone, en cambio, de métodos que para clasificar los individuos, se basan
en métodos de aprendizaje diseñados con la cuantificación en mente. La segunda
clase es la de métodos no agregativos, es decir, métodos que resuelven la tarea
de cuantificación “holísticamente”, es decir, sin clasificar a los individuos.
La idea de esta tésis no es la de mostrar todos los métodos propuestos hasta la
actualidad, sino la de mencionar a continuación los métodos más populares.

\section{Métodos Agregativos}\label{puntual:agregativos}

\subsection{Con clasificadores generales}

Dentro de los métodos agregativos, algunos de ellos requieren como entrada las
etiquetas de clases predichas (es decir, clasificacores duros), mientras que
otros requieren como entrada las probabilidades {\it a posteriori\/} de
pertenencia a cada clase (es decir, clasificacores blandos)\footnote{Los
clasificadores blandos se pueden convertir en duros usando umbrales de
clasificación}. En estos últimos, además, las probabilidades {\it a
posteriori\/} deben estar calibradas (para mayor información sobre calibración
consultar el Apéndice~\ref{appendix:calibracion}).

\subsection{Con clasificadores generales}

\subsection{Con clasificadores específicos}

\section{Métodos No Agregativos}
