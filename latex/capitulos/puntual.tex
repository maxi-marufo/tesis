\chapter{Estimación Puntual}

Durante los últimos años, se han propuesto varios métodos de cuantificación
desde diferentes perspectivas y con diferentes objetivos. En términos generales,
se pueden distinguir dos grandes clases de métodos en la literatura. La primera
clase es la de métodos agregativos, es decir, métodos que requieren la
clasificación de todos los individuos como un paso intermedio. Dentro de los
métodos agregativos, se pueden identificar dos subclases. La primera subclase
incluye métodos basados en clasificadores de propósito general; en estos métodos
la clasificación de los elementos individuales realizados como un paso
intermedio puede lograrse mediante cualquier clasificador. La segunda subclase
se compone, en cambio, de métodos que para clasificar los individuos, se basan
en métodos de aprendizaje diseñados con la cuantificación en mente. La segunda
clase es la de métodos no agregativos, es decir, métodos que resuelven la tarea
de cuantificación “holísticamente”, es decir, sin clasificar a los individuos.
La idea de esta tésis no es la de mostrar todos los métodos propuestos hasta la
actualidad, sino la de mencionar a continuación los métodos más populares.

\section{Métodos Agregativos}\label{puntual:agregativos}

\subsection{Con clasificadores generales}

Dentro de los métodos agregativos, algunos de ellos requieren como entrada las
etiquetas de clases predichas (es decir, clasificacores duros), mientras que
otros requieren como entrada las probabilidades {\it a posteriori\/} de
pertenencia a cada clase (es decir, clasificacores blandos)\footnote{Los
clasificadores blandos se pueden convertir en duros usando umbrales de
clasificación}. En estos últimos, además, las probabilidades {\it a
posteriori\/} deben estar calibradas (para mayor información sobre calibración
consultar el Apéndice~\ref{appendix:calibracion}).

\subsubsection{Clasificar y Contar (CC)}

El método más sencillo y directo para construir un cuantificador para
clasificación (tanto binaria como multiclase) es aplicar el enfoque {\it
Classify \& Count\/}~\cite{forman2005counting}. {\it CC\/} juega un papel
importante en la investigación de cuantificación ya que siempre se utiliza como
el {\it baseline\/} que cualquier método de cuantificación razonable debe
mejorar. Este método consiste simplemente en: (i) ajustar un clasificador duro,
y luego (ii), utilizando dicho clasificador, clasificar las instancias de la
muestra de prueba, contando la proporción de cada clase. Generalizando el
estimador de {\it CC\/} para el caso multiclase, el mismo queda entonces
definido por:

\begin{equation}
    \hat p^{CC}_{U}(y) = \frac{|\{\mathbf{x} \in U|h(\mathbf{x})=y\}|}{|U|}
\end{equation}

Donde se uso $U$ para denotar el conjunto de datos de evaluación (no
etiquetados, o {\it unlabeled\/}) y $h$ para la función de decisión del
clasificador duro.

Es evidente que podemos obtener un cuantificador perfecto si el clasificador es
también perfecto. El problema es que obtener un clasificador perfecto es casi
imposible en aplicaciones reales, y luego el cuantificador hereda el sesgo del
clasificador. Este aspecto se analiza en varios artículos tanto desde una
perspectiva teórica como práctica, como lo
hizo~\citeauthor{forman2008quantifying}, y como también ya lo hemos mencionado
en~\ref{problema:clasificar_y_contar}.

\subsection{Con clasificadores específicos}

\section{Métodos No Agregativos}
