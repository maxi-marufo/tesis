\chapter{Estimación Puntual}

Durante los últimos años, se han propuesto varios métodos de cuantificación
desde diferentes perspectivas y con diferentes objetivos. En términos generales,
se pueden distinguir dos grandes clases de métodos en la literatura. La primera
clase es la de métodos agregativos, es decir, métodos que requieren la
clasificación de todos los individuos como un paso intermedio. Dentro de los
métodos agregativos, se pueden identificar dos subclases. La primera subclase
incluye métodos basados en clasificadores de propósito general; en estos métodos
la clasificación de los elementos individuales realizados como un paso
intermedio puede lograrse mediante cualquier clasificador. La segunda subclase
se compone, en cambio, de métodos que para clasificar los individuos, se basan
en métodos de aprendizaje diseñados con la cuantificación en mente. La segunda
clase es la de métodos no agregativos, es decir, métodos que resuelven la tarea
de cuantificación “holísticamente”, es decir, sin clasificar a los individuos.
Aquí de nuevo se hará tratarán métodos destinados a la cuantificación binaria,
aunque como ya se mencionó la mayoría de ellos pueden luego extenderse a
problemas multiclase.

\section{Métodos Agregativos}

Dentro de los métodos agregativos, algunos de ellos requieren como entrada las
etiquetas de clases predichas (es decir, clasificacores duros), mientras que
otros requieren como entrada las probabilidades {\it a posteriori\/} de
pertenencia a cada clase (es decir, clasificacores blandos).

\subsection{Con clasificadores generales}

\subsection{Con clasificadores específicos}

\section{Métodos No Agregativos}
