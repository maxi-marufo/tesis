\chapter{Introducción}\label{introduccion}

\section{Marco teórico}\label{introduccion:marco_teorico}

La tarea de cuantificación (conocida en inglés como {\it quantification\/})
consiste en proporcionar predicciones agregadas para conjuntos de datos, en vez
de predicciones particulares sobre los datos individuales. Por ejemplo, para el
caso de la cuantificación aplicada a la clasificación, se busca predecir la
proporción de clases de un conjunto de individuos, en vez de la clase particular
de cada individuo. Para ello, se aplican modelos que se ajustan usando datos de
entrenamiento cuya distribución puede ser distinta a la de los datos de
prueba~\cite{forman2005counting}.

Si bien en principio no es necesario realizar predicciones por cada individuo,
muchos de los métodos se basan en obtener la cuantificación de esa manera, ya
que hacer predicciones individuales suele ser un requisito de por sí de las
aplicaciones prácticas, o porque ya existen en ellas modelos que las generen.
Además, cabe aclarar que si bien la aplicación más popular es con respecto a
tareas de clasificación (sobre las cuales basaremos este trabajo, y en
particular, sobre clasificación binaria), también se puede aplicar
cuantificación a problemas de regresión, ordinalidad, etc.

Un ejemplo práctico puede ser predecir la proporción de comentarios a favor o en
contra sobre un producto, servicio o candidato en una red social. En este caso,
se puede utilizar un clasificador para predecir por cada comentario si la
opinión es positiva (o negativa), y luego obtener la proporción de comentarios a
favor contándolos y dividiéndolos por el total (este método es el más simple y
es conocido como {\it Classify \& Count\/} o {\it CC\/}).

Si hablamos entonces de cuantificación binaria, se tiene que por cada muestra $i
\in \{1,\dots,n\}$, $(\boldsymbol{X}_i,Y_i,S_i)$ es un vector de variables
aleatorias tal que $\boldsymbol{X}_i \in \mathbb{R}^d$ son las características
de la muestra, $Y_i \in C$ con $C=\{1,0\}$ indica la clase a la que pertenece y
$S_i \in \{1,0\}$ indica si fue etiquetada (y pertenece entonces al conjunto de
entrenamiento) o no. Es decir, cuando $S_i=0$, entonces $Y_i$ no es observable.
El objectivo es estimar $\theta:= \mathbb{P}(Y=1|S=0)$\footnote{En
cuantificación, se lo nombra generalmente como $p$ (o $p_1,\dots,p_{\#C}$ o
$p(c)$ para el caso multiclase) en vez de $\theta$, por lo que en este trabajo
también se usará esta nomenclatura.}, es decir, la prevalencia de etiquetas
positivas entre muestras no etiquetadas. Esta prevalencia no se asume de ser la
misma que en las muestras etiquetadas, $\mathbb{P}(Y=1|S=1)$. Además, el
estimador de $\theta$ debe depender sólo de los datos disponibles, es decir, de
las características de todas las muestras y de las etiquetas que fueron
obtenidas. Los supuestos que se asumen~\cite{vaz2019quantification} son:

\begin{itemize}
  \item $(\boldsymbol{X}_1,Y_1,S_1) \dots (\boldsymbol{X}_n,Y_n,S_n)$ son
  independientes
  \item Por cada $s \in \{0,1\}$,
  $(\boldsymbol{X}_1,Y_1)|S_1=s,\dots,(\boldsymbol{X}_n,Y_n)|S_n=s$ son
  idénticamente distribuidas.
  \item Por cada $(y_1,\dots,y_n)\in{\{0,1\}}^n$,
  $(\boldsymbol{X}_1,\dots,\boldsymbol{X}_n)$ es independiente de
  $(S_1,\dots,S_n)$ condicionado a $(Y_1,\dots,Y_n)=(y_1,\dots,y_n)$
\end{itemize}

Usando la distribución de probabilidad conjunta, podemos factorizar usando las
distribuciones condicionales:
\begin{equation}
    \mathbb{P}(\boldsymbol{X},Y,S)=\mathbb{P}(\boldsymbol{X}|Y,S)\mathbb{P}(Y|S)\mathbb{P}(S)
\end{equation}
Luego, usando el tercer supuesto mencionado, podemos
hacer~\cite{moreno2012unifying}:
\begin{equation}
    \mathbb{P}(\boldsymbol{X},Y,S)=\mathbb{P}(\boldsymbol{X}|Y)\mathbb{P}(Y|S)\mathbb{P}(S)
\end{equation}
Si bien existen varios métodos propuestos para el aprendizaje de
cuantificación~\cite{esuli2023learning, gonzalez2017review}, el mismo es todavía
relativamente desconocido incluso para expertos en aprendizaje automático. La
razón principal es la creencia errónea de que es una tarea trivial que se puede
resolver usando un método directo, como {\it CC}. La cuantificación requiere
métodos más sofisticados si el objetivo es obtener modelos óptimos, y su
principal dificultad radica en la definición del problema, ya que las
distribuciones de los datos de entrenamiento y de prueba pueden ser distintas.
Por ejemplo, si la diferencia entre $\mathbb{P}(Y=1|S=0)$ y
$\mathbb{P}(Y=1|S=1)$ es grande, los métodos simples como {\it CC\/} suelen
tener bajo rendimiento.

Un método de cuantificación muy popular en la literatura y que sí se adapta a
los cambios entre $\mathbb{P}(Y=1|S=0)$ y $\mathbb{P}(Y=1|S=1)$ es el propuesto
por~\citet{saerens2002adjusting}, conocido como {\it Expectation Maximization
for Quantification\/} -{\it EMQ\/}- o {\it SLD\/} por las siglas de sus autores.
El mismo es un método iterativo que corrige, mediante el Teorema de Bayes, las
predicciones de probabilidad de pertenencia a las clases dadas por el modelo de
clasificación ya ajustado (sin necesidad de reajuste), y como consecuencia
estima también la proporción de clases en la muestra de prueba. Este método es
una aplicación directa del algoritmo de Esperanza-Maximización {\it -EM-}, y se
puede probar que maximiza la verosimilitud en los datos de prueba. Se ha
estudiado también que el método {\it EMQ\/} mejora aún más las predicciones de
cuantificación si el clasificador utilizado está
calibrado~\cite{esuli2020critical, alexandari2020maximum}, es decir, si sus
predicciones de probabilidad asociadas a las clases predichas representan la
probabilidad real de pertenencia a las clases~\cite{guo2017calibration}.

Por otro lado, muy pocos son los trabajos sobre la construcción de intervalos de
confianza y predicción en cuantificación~\cite{tasche2019confidence}. La mayoría
de ellos se basan en emplear los métodos de predicción puntual junto con la
técnica de {\it bootstrapping\/}~\cite{hopkins2010method,
daughton2020constructing, daughton2021bootstrapping}, que puede ser
computacionalmente costosa en este tipo de tareas, o en métodos
asintóticos~\cite{vaz2019quantification}, que no funcionan bien con tamaños de
muestra pequeños y además requieren estimar la varianza. Dentro de los trabajos
sobre intervalos de confianza o predicción en cuantificación aplicada a la
clasificación (no se encontraron de hecho trabajos sobre intervalos en otro tipo
de problemas), Keith y O'Connor~\cite{keith2018uncertainty} proponen dos
métodos:

\begin{itemize}
    \item El {\it baseline}, al que llaman {\it PB-PCC}, es un método asintótico
    basado en la distribución Poisson-Binomial~\cite{le1960approximation,
    wang1993number}, donde proponen utilizar la media y varianza conocida para
    esta distribución para calcular el intervalo mediante la distribución
    normal. Existen tres problemas en el método propuesto en el trabajo:
        \begin{enumerate}
            \item Se basa en el método de cuantificación de predicción puntual
            conocido como {\it Probabilistic Classify \& Count -PCC-}, que no
            suele tener muy buenos resultados ya que no se ajusta a las
            diferencias entre $\mathbb{P}(Y=1|S=0)$ y
            $\mathbb{P}(Y=1|S=1)$.\label{keith_item_uno}
            \item En el trabajo no se verifica que el clasificador esté
            calibrado, lo que podría degradar los resultados del
            cuantificador.\label{keith_item_dos}
            \item Los métodos asintóticos no son buenos con muestras pequeñas.
        \end{enumerate}
    \item Su propuesta de mejora se basa en computar la verosimilitud marginal
    sobre $\theta$, la proporción de clases en la población de prueba, para
    obtener la distribución {\it a posteriori\/} de $\theta$. Luego, para
    obtener la predicción puntual se obtiene el máximo de la distribución. Es
    decir, que al igual que el método {\it EMQ}, se busca maximizar la
    verosimilitud, pero en este caso sin utilizar el algoritmo {\it EM\/} sino
    de forma directa. Para obtener en cambio los intervalos, se proponen dos
    opciones, la primera es usar la aproximación asintótica y la segunda es
    construir una grilla para la distribución {\it a posteriori\/} de $\theta$.
    Este método es bastante efectivo. Sin embargo, hay que tener en cuenta que
    aquí se estima la proporción de clases en la población de prueba, y no en la
    muestra de prueba.
\end{itemize}

~\citet{denham2021gain} propone un método que asume condiciones más generales de
cambio en las distribuciones entre los datos de entrenamiento y de prueba, ya
que no asume la tercera suposición mencionada anteriormente. El método que
proponen tiene, sin embargo, peor desempeño cuando esa condición sí se cumple.
No obstante, es interesante resaltar los dos métodos {\it baseline\/} que
utilizan para comparar con su propuesta de mejora:

\begin{itemize}
    \item El primer {\it baseline\/} es también basado en {\it PB-PCC\/} y
    similar al propuesto en Keith y O'Connor. La diferencia en este trabajo es
    que en vez de usar la aproximación asintótica, computan la función de
    distribución exacta mediante el método propuesto
    por~\citet{hong2013computing}, que utiliza la transformada rápida de Fourier
    ({\it FFT\/}) para hacerlo de forma eficiente. Este método sigue teniendo
    los primeros dos problemas mencionados en~\ref{keith_item_uno}
    y~\ref{keith_item_dos}.
    \item El segundo {\it baseline\/} es muy similar a la propuesta de mejora de
    Keith y O'Connor. La única diferencia es que usa el algoritmo de {\it EM\/}
    en vez de hacer el cálculo directo para obtener el máximo de la distribución
    {\it a posteriori}.
\end{itemize}

\section{Propuesta}\label{introduccion:propuesta}

Se propone un método para la elaboración de intervalos de predicción de la
proporción de clases en muestras de prueba sin etiquetar, a partir de un
conjunto de datos con etiquetas conocidas (conjunto de entrenamiento). Los pasos
son:

\begin{enumerate}
    \item Ajustar un modelo de clasificación con los datos de entrenamiento
    \item Aplicar un método de calibración al clasificador para crear buenos
    estimadores de probabilidad para cada individuo.
        \begin{itemize}
            \item Se compararon todos los métodos de calibración mencionados en
            la bibliografía, evaluados bajo el problema de cuantificación,
            obteniendo los mejores resultados con los métodos propuestos
            en~\citet{alexandari2020maximum}.
        \end{itemize}
    \item Aplicar el método de cuantificación de estimación puntual de
    preferencia.\label{propuesta_item_tres}
        \begin{itemize}
            \item Se compararon los principales métodos mencionados en la
            bibliografía, obteniendo los mejores resultados con los métodos de
            {\it EMQ\/} y {\it PACC}.
        \end{itemize}
    \item Aplicar el paso de maximización de la esperanza propuesto
    en~\citet{saerens2002adjusting} para ajustar las predicciones de
    probabilidad de cada individuo en base a la predicción puntual de la
    proporción de clases obtenidas con el método elegido
    en~\ref{propuesta_item_tres}.
    \item Aplicar el método propuesto por~\citet{hong2013computing} para obtener
    la distribución exacta de la proporción de clases en la muestra de
    prueba.\label{propuesta_item_cinco}
    \item Utilizar la distribución obtenida en~\ref{propuesta_item_cinco} para
    elaborar un intervalo de predicción exacto de la proporción de clases en la
    muestra con el nivel que corresponda.
\end{enumerate}

Se presenta el método tanto de forma teórica como su evaluación empírica. Se
elaboraron simulaciones para su evaluación y comparación con los métodos
propuestos en Keith y O'Connor~\cite{keith2018uncertainty},
en~\citet{denham2021gain} y con los intervalos obtenidos mediante {\it
bootstrapping}.
