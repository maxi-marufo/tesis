\documentclass[a4paper, twoside, 11pt, spanish]{article}

%% Paquetes
\usepackage[es-tabla]{babel}
\usepackage{amsmath, amsfonts, amssymb, amsthm}
\usepackage{color, graphicx, rotating, subcaption}
\usepackage[margin=2.2cm]{geometry}
\usepackage{hyperref, url}
\usepackage[font=small, labelfont=bf]{caption}
\usepackage{enumerate, enumitem}
\usepackage{tabularx}
\usepackage{ctable}
\usepackage{multicol}

\usepackage[square,numbers,sort&compress]{natbib}

\renewcommand{\baselinestretch}{1.2} % interlineado
\decimalpoint{} % cambia coma por punto

%% Bibliografia
\bibliographystyle{unsrtnat}
\addto{\captionsspanish}{\renewcommand{\bibname}{Referencias}}

%\begin{document}

\begin{titlepage}

\title{ \textbf{Proyecto de tesis para optar al título de \\ Magister en
    Estadística Matemática}\\[2.5ex]
    \textit{Estimación de proporción de clases en muestras no etiquetadas \\ 
    mediante modelos de cuantificación} }

\author{ \textbf{Director:} Dr. Andrés Farall \\[2.5ex]
    \textbf{Alumno:} Ing. Maximiliano Marufo da Silva \\[2.5ex]
    \normalsize{Facultad de Ciencias Exactas y Naturales} \\
    \normalsize{Universidad de Buenos Aires} \\
}
\date{}

\end{titlepage}

\maketitle

\section*{Tema de investigación sobre el cual tratará el trabajo}

La tarea de cuantificación consiste en proporcionar predicciones agregadas para
conjuntos de datos, en vez de predicciones particulares sobre los datos
individuales. Por ejemplo, para el caso de la cuantificación aplicada a la
clasificación, se busca predecir la proporción de clases de un conjunto de
individuos, en vez de la clase particular de cada individuo. Un ejemplo práctico
puede ser predecir la proporción de comentarios a favor o en contra sobre un
producto, servicio o candidato en una red social. En este caso, se puede
utilizar un clasificador para predecir por cada comentario si la opinión es
positiva (o negativa), y luego obtener la proporción de comentarios a favor
contándolos y dividiéndolos por el total. Sin embargo, esta estrategia es
subóptima: si bien un clasificador perfecto es también un cuantificador
perfecto, un buen clasificador puede ser un mal cuantificador.

En cuantificación se aplican modelos que se ajustan usando datos de
entrenamiento cuya distribución puede ser distinta a la de los datos de
prueba~\cite{forman2005counting}. Si hablamos entonces de cuantificación
binaria, se tiene que por cada muestra $i \in \{1,\dots,n\}$,
$(\boldsymbol{X}_i,Y_i,S_i)$ es un vector de variables aleatorias tal que
$\boldsymbol{X}_i \in \mathbb{R}^d$ son las características de la muestra, $Y_i
\in C$ con $C=\{1,0\}$ indica la clase a la que pertenece y $S_i \in \{1,0\}$
indica si fue etiquetada (y pertenece entonces al conjunto de entrenamiento) o
no. Es decir, cuando $S_i=0$, entonces $Y_i$ no es observable. El objetivo es
estimar $\theta:= \mathbb{P}(Y=1|S=0)$, es decir, la prevalencia de etiquetas
positivas entre muestras no etiquetadas. Esta prevalencia no se asume de ser la
misma que en las muestras etiquetadas, $\mathbb{P}(Y=1|S=1)$. Además, el
estimador de $\theta$ debe depender solo de los datos disponibles, es decir, de
las características de todas las muestras y de las etiquetas que fueron
obtenidas. Los supuestos que se asumen~\cite{vaz2019quantification} son:

\begin{itemize}
  \item $(\boldsymbol{X}_1,Y_1,S_1) \dots (\boldsymbol{X}_n,Y_n,S_n)$ son
  independientes
  \item Por cada $s \in \{0,1\}$,
  $(\boldsymbol{X}_1,Y_1)|S_1=s,\dots,(\boldsymbol{X}_n,Y_n)|S_n=s$ son
  idénticamente distribuidas.
  \item Por cada $(y_1,\dots,y_n)\in{\{0,1\}}^n$,
  $(\boldsymbol{X}_1,\dots,\boldsymbol{X}_n)$ es independiente de
  $(S_1,\dots,S_n)$ condicionado a $(Y_1,\dots,Y_n)=(y_1,\dots,y_n)$
\end{itemize}

\section*{Antecedentes sobre el tema}

Si bien existen varios métodos propuestos para el aprendizaje de
cuantificación~\cite{esuli2023learning, gonzalez2017review}, el mismo es todavía
relativamente desconocido incluso para expertos en aprendizaje automático. La
razón principal es la creencia errónea de que es una tarea trivial que se puede
resolver usando un método directo como el de clasificar y contar. La
cuantificación requiere métodos más sofisticados si el objetivo es obtener
modelos óptimos, y su principal dificultad radica en la definición del problema,
ya que las distribuciones de los datos de entrenamiento y de prueba pueden ser
distintas.

Aunque en principio no es necesario realizar predicciones por cada individuo,
muchos de los métodos se basan en obtener la cuantificación de esa manera, ya
que hacer predicciones individuales suele ser un requisito de por sí de las
aplicaciones prácticas, o porque ya existen en ellas modelos que las generen.
Además, cabe aclarar que, si bien la aplicación más popular es con respecto a
tareas de clasificación (sobre las cuales basaremos principalmente este trabajo,
y en particular, sobre clasificación binaria), también se puede aplicar
cuantificación a problemas de regresión, ordinalidad, etc.

La literatura sobre métodos relacionados con cuantificación está un tanto
desconectada. Algunos de los métodos que pueden usarse como cuantificadores han
sido ideados para otros fines, principalmente para mejorar la precisión en
clasificación cuando cambia el dominio. El desempeño de este último grupo ha
sido normalmente estudiado solo en términos de mejora en las tareas de
clasificación pero no como cuantificadores. Dado este escenario, y debido a la
variedad de campos en los que ha surgido como una necesidad de aplicación, los
algoritmos que se pueden aplicar para tareas de cuantificación aparecen en
artículos que usan diferentes palabras clave y nombres, como {\it
counting\/}~\cite{lewis1995evaluating}, {\it prior probability
shift\/}~\cite{moreno2012unifying, storkey2009training}, {\it posterior
probability estimation\/}~\cite{alaiz2011class}, {\it class prior
estimation\/}~\cite{du2014class, chan2006estimating, zhang2010transfer}, {\it
class prior change\/}~\cite{du2014semi}, {\it prevalence
estimation\/}~\cite{barranquero2013study}, {\it class ratio
estimation\/}~\cite{asoh2012fast} o {\it class distribution
estimation\/}~\cite{gonzalez2013class, limsetto2011handling,
xue2009quantification}, por citar solo algunos de ellos.

\section*{Naturaleza del aporte proyectado}

El objetivo de este trabajo es describir el problema en el que se enmarca la
cuantificación, justificando las razones de por qué es necesario utilizar
modelos optimizados para estos casos, y resumir el estado del arte en el área,
evaluando mediante simulaciones los principales modelos propuestos.

\section*{Metodología tentativa a seguir para lograr los objetivos propuestos}

En junio del 2022 el maestrando comenzó elaborando un estudio en profundidad del
tema, que incluyó lectura exhaustiva de bibliografía, realización de
simulaciones e incluso el inicio de la redacción del informe para la tesis. Con
lo cual el presente plan se presenta luego de ya haber comenzado con el estudio
y elaboración del trabajo. Los pasos que restan para lograr los objetivos
propuestos, junto con sus duraciones estimadas, son:

\begin{enumerate}
  \item Finalización de la redacción de la parte teórica del informe (2 meses).
  \item Ajuste y adecuación de las simulaciones ya realizadas (2 semanas).
  \item Elaboración de conclusiones, revisión final y correcciones (2 semanas).
\end{enumerate}

\bigskip

\bibliography{references}

\end{document}