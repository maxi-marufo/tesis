\begin{center}
  \textbf{\Large Resumen}\label{resumen}
\end{center}

La cuantificación consiste en proporcionar predicciones agregadas para conjuntos
de datos, en lugar de predicciones individuales para cada dato. En el contexto
de la clasificación, esto se traduce en predecir la proporción de cada clase
dentro de un conjunto de instancias, en lugar de la clase particular de cada
instancia individualmente.

Un ejemplo práctico es la predicción de la proporción de comentarios positivos y
negativos sobre un producto, servicio o candidato en redes sociales. Si bien se
podría utilizar un clasificador para predecir el sentimiento de cada comentario
y, posteriormente, derivar las proporciones de clase, esta estrategia es
subóptima y a menudo produce estimaciones sesgadas de la prevalencia, lo que
resulta en una baja precisión en la cuantificación. Por consiguiente, se han
desarrollado métodos específicos para abordar la cuantificación como una tarea
independiente.

Los modelos de cuantificación se entrenan con datos cuya distribución puede
diferir de la de los datos de prueba. En el contexto de la cuantificación
binaria, para cada instancia $i \in \{1,\dots,n\}$, consideramos un vector de
variables aleatorias $(\boldsymbol{X}_i,Y_i,S_i)$, donde $\boldsymbol{X}_i \in
\mathbb{R}^d$ representa las características de la instancia, $Y_i \in \{0, 1\}$
denota su etiqueta de clase, y $S_i \in \{0, 1\}$ indica si la instancia está
etiquetada (y, por lo tanto, pertenece al conjunto de entrenamiento). Cuando
$S_i = 0$, la etiqueta $Y_i$ no es observable. El objetivo es estimar $\theta :=
\mathbb{P}(Y=1|S=0)$, es decir, la prevalencia de etiquetas positivas entre las
instancias no etiquetadas. No se asume que esta prevalencia sea igual a la de
las instancias etiquetadas, $\mathbb{P}(Y=1|S=1)$. Además, el estimador de
$\theta$ debe depender únicamente de los datos disponibles: las características
de todas las instancias y las etiquetas observadas.

El objetivo de este trabajo es describir el problema de la cuantificación,
justificando la necesidad de utilizar modelos optimizados para estos casos, y
presentar una revisión del estado del arte en este campo, evaluando mediante
simulaciones los principales modelos propuestos.

\bigskip

\textbf{Palabras Clave:} Cuantificación, estimación de proporción de clases,
cambio de distribución