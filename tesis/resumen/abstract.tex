\begin{center}
  \textbf{\Large Abstract}\label{abstract}
\end{center}

Quantification aims to provide aggregate predictions for datasets, rather than
individual predictions for each data point. In the context of classification,
this translates to predicting the proportion of each class within a set of
instances, rather than the specific class of each instance individually.

A practical example is predicting the proportion of positive and negative
comments regarding a product, service, or candidate on social media. While a
classifier could be used to predict the sentiment of each comment and
subsequently derive class proportions, this strategy is suboptimal and often
yields biased prevalence estimates, resulting in poor quantification accuracy.
Consequently, dedicated methods have been developed to address quantification as
an independent task.

Quantification models are trained on data whose distribution may differ from
that of the test data. In the context of binary quantification, for each
instance $i \in \{1,\dots,n\}$, we consider a vector of random variables
$(\boldsymbol{X}_i,Y_i,S_i)$, where $\boldsymbol{X}_i \in \mathbb{R}^d$
represents the instance's features, $Y_i \in \{0, 1\}$ denotes its class label,
and $S_i \in \{0, 1\}$ indicates whether the instance is labeled (and therefore
belongs to the training set). When $S_i = 0$, the label $Y_i$ is unobserved. The
objective is to estimate $\theta := \mathbb{P}(Y=1|S=0)$, i.e., the prevalence
of positive labels among unlabeled instances. This prevalence is not assumed to
be equal to that of the labeled instances, $\mathbb{P}(Y=1|S=1)$. Furthermore,
the estimator of $\theta$ must depend solely on the available data: the features
of all instances and the observed labels.

This work aims to describe the quantification problem, justifying the need for
optimized models in these cases, and to present a review of the state of the art
in this field, evaluating the main proposed models through simulations.

\bigskip

\textbf{Keywords:} Quantification, class proportion estimation, distribution
shift